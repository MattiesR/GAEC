\documentclass[a4paper,10pt]{article}
\usepackage[utf8]{inputenc}

\usepackage[english]{babel}
\usepackage{xcolor}
\usepackage[compact,small]{titlesec}
\usepackage{booktabs}
\usepackage{multirow}
\usepackage{amsfonts,amsmath,amssymb}
\usepackage{marginnote}
\usepackage[top=1.8cm, bottom=1.8cm, outer=1.8cm, inner=1.8cm, heightrounded, marginparwidth=2.5cm, marginparsep=0.5cm]{geometry}
\usepackage{enumitem}
\setlist{noitemsep,parsep=2pt}
\newcommand{\highlight}[1]{\textcolor{kuleuven}{#1}}
\usepackage{pythonhighlight}
\usepackage{cleveref}
\usepackage{graphicx}

\newcommand{\nextyear}{\advance\year by 1 \the\year\advance\year by -1}
\newcommand{\thisyear}{\the\year}
\newcommand{\deadlineGroup}{November 7, \thisyear{} at 9:00 CET}

\newcommand{\ReplaceMe}[1]{{\color{blue}#1}}
\newcommand{\RemoveMe}[1]{{\color{purple}#1}}

\setlength{\parskip}{5pt}

%opening
\title{\vspace{-2cm}Evolutionary Algorithms: Group report}
\author{\ReplaceMe{Group Member 1}, \ReplaceMe{Group Member 2}, and \ReplaceMe{Group Member 3}}

\begin{document}
\fontfamily{ppl}
\selectfont{}

\maketitle

%%% You can remove the Formal requirements section
\RemoveMe{
\section*{Formal requirements}

Please respect the structure of this template. You can remove the instructions in this section from your report. The blue text should be replaced with your discussion. Your report can be \textbf{at most $4$ pages} long. 

It is recommended that you use this \LaTeX{} template, but you are allowed to reproduce it with the same structure in a WYSIWYG-editor. You should replace the blue text with your discussion. The questions we ask in blue are there to help you decide which topics to discuss, rather than an exact list of questions that must be answered.

This report should be uploaded to Toledo by \deadlineGroup. It must be in the \textbf{Portable Document Format} (pdf) and must be named \texttt{r0123456\_intermediate.pdf}, where r0123456 should be replaced with your student number. \textbf{Each group member should hand it in individually on Toledo.}
}

\section{A basic evolutionary algorithm \hfill(target: $1$ page)} 

\subsection{Representation}

\ReplaceMe{How do you represent the candidate solutions? What is your main motivation to choose this one? What other options did you consider? How did you implement this specifically in Python (e.g., a list, set, numpy array, etc)?}

\subsection{Initialization}

\ReplaceMe{How do you initialize the population? How do you generate random cycles?}

\subsection{Selection operators}

\ReplaceMe{Which selection operators did you consider? If they are not from the slides, describe them. Which one did you implement? Can you motivate why you chose this one? Are there parameters that need to be chosen? What do you think are reasonable parameter values?}

\subsection{Mutation operator}

\ReplaceMe{Which mutation operator(s) did you consider and implement? If they are not from the slides, describe them. Which one did you implement? Why did you choose that one specifically? Do you believe it will introduce sufficient randomness? Can that be controlled with parameters?}

\subsection{Recombination operator}

\ReplaceMe{Which recombination operator(s) did you consider? If they are not from the slides, describe them. Which one did you implement? Why did you choose that one specifically? Explain how you believe that this operator can produce offspring that combine the best features from their parents. How does your operator behave if there is little overlap between the parents? Can your recombination be controlled with parameters; what behavior do they change? Do you use self-adaptivity?}

\subsection{Elimination operators}

\ReplaceMe{Which elimination operators did you consider? If they are not from the slides, describe them. Which one did you implement? Can you motivate why you chose this one? Are there parameters that need to be chosen? What do you think are reasonable parameter values?} 

\subsection{Stopping criterion}

\ReplaceMe{When do you think the evolutionary algorithm should stop? Which stopping criterion did you implement? Did you combine several criteria?}

\section{Implementation generation \hfill(target: $1$ page)}

\subsection{Prompts to generate code}
\ReplaceMe{How did you approach the code generation: multiple short prompts to generate individual functions, or did you ask for the whole implementation at once? Which was more efficient? Did you seed Copilot with information before asking it to generate code? What information did you give it? Record the prompts that you used to generate the code. Did you apply some postprocessing to the generated code? What did you do?}

\subsection{Prompts to fix errors}
\ReplaceMe{Was the generated code free of errors? If not, what was wrong? What type of errors were they: misinterpretation, logical, or programming errors? How did you instruct Copilot to fix these errors? Record the prompts. Did you apply some postprocessing? What did you do?}

\subsection{Critical reflection}
\ReplaceMe{Provide your critical reflection on the process of using Copilot to generate a basic evolutionary algorithm implementation. What were the positive and negative points? Was it more or less efficient than writing code from scratch? Was the debugging more or less efficient using Copilot? Do you think the descriptions you provided in section 1 could be used as effective prompts? Why? Would you recommend the use of Copilot for exercise session 2? Any other critical thoughts?}


\section{Numerical experiments \hfill(target: $1$ page)}

\subsection{Chosen parameter values}

\subsubsection{Parameters}
\textbf{Population size}, \textbf{Number of generations}, \textbf{Selection method} (tournament selection with tournament size $k$), \textbf{Crossover rate}, \textbf{Adaptive mutation rate} (base value, maximum value, and increase factor), and \textbf{Elitism} (number of elite individuals to keep).

\subsubsection{Parameter Selection Flow}
In general, the parameter selection flow was an iterative approach changing parameters based on previous knowledge from the lectures and educated guesses. Letting the algorithm run with a given set of parameters, evaluating the results and optimizing parameters in a specific way to change the behavior of our algorithm to provide better results. For example, decrease selection pressure to converge slower. 

We started with educated guesses on initial parameters, comparing the results against the given baseline. We immediately increased “Max. Generations" No improvement generations to let our algorithm explore longer.  

We experimented a lot with different values and value combinations to fine-tune the behavior of our algorithm: 
\begin{itemize}
    \item Decreasing population size helped us to speed up the process and pave the way for local optimization techniques (which scale linearly with population size)
    \item Because we had the feeling that our algorithm didn’t do enough exploration, we changed mutation rates, k-tournament selection rounds and elite counts. 
\end{itemize}

In the end Table \ref{chosen_params} lists the parameters that seemed to return the best results, surpassing the baseline.


\begin{table}
\centering    
\begin{tabular}{l|l}
\textbf{Parameter}               & \textbf{Value} \\ \hline
Population size                  & 200              \\
Max generations                  & 5000           \\
No improvement generations      & 300             \\
Selection method (tournament k)  & 5               \\
Adaptive mutation rate (base)    & 0.3            \\
Adaptive mutation rate (max)     & 0.9             \\
Adaptive mutation rate (factor)  & 1.05            \\
Elitism (number of elite indiv.) & 2             
\end{tabular}
\caption{Chosen parameter values for the evolutionary algorithm}    
\label{chosen_params}
\end{table}


%%%%%%
\subsection{Preliminary results}

In total we executed the algorithm with the chosen parameters from Table \ref{chosen_params} on 100 different random seeds. Figure \ref{best_run_convergence} shows a convergence graph of the best run.
The \textbf{best value} we achieved was \textbf{13844.3383}.
We do not think this is the global optimum, as there is no proof of optimality and there is always a slight chance that a better solution exists.

When solving this problem on 100 different random seeds, we observed some variation in the best solution.
Figure \ref{histogram_best_values} shows a histogram of the best values found over the 100 runs. It shows a mean of 14110.64 with a standard deviation of 179.17. This is definitely a significant variation, indicating that the algorithm's performance is still sensitive to the initial random seed. (But at least its a little scewed towards lower values).
Figure \ref{all_runs_best_value} shows the convergence of all 100 runs.


The program ran for for apprximately 17-20 seconds until convergence and stops after 300 generations of no improvement.
This is not the fastest runtime but we wanted to use the given time of 5 min to find a very good solution.
It converges relatively quickly (within 300-400 generations). Figure \ref{best_run_convergence_zoom} shows nicely how the adaptive mutation rate forces exploration after some time of stagnation.
Memory usage is not optimized but should be fine for all problem sizes as population management and individual representation are done efficiently.
The mean value for the whole population being way higher then the best member could indicate a high population diversity (Figure \ref{best_run_convergence}). 


\begin{figure}[h!]
    \centering
    \includegraphics[width=0.8\textwidth]{images/mean_best_value_over_time.png} 
    \caption{Best run convergence graph over time.}
    \label{best_run_convergence}
\end{figure}

\begin{figure}[h!]
    \centering
    \includegraphics[width=0.8\textwidth]{images/mean_best_value_over_time_zoomed.png} 
    \caption{Zoomed in version of Figure \ref{best_run_convergence}.}
    \label{best_run_convergence_zoom}
\end{figure}

\begin{figure}[h!]
    \centering
    \includegraphics[width=0.8\textwidth]{images/histogram_best_values.png} 
    \caption{Histogram of best values for 100 different random seeds.}
    \label{histogram_best_values}
\end{figure}

\begin{figure}[h!]
    \centering
    \includegraphics[width=0.8\textwidth]{images/best_value_over_iterations.png} 
    \caption{Best value for 100 different random seeds.}
    \label{all_runs_best_value}
\end{figure}


\end{document}
